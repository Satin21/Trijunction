\chapter{Conclusions}
\begin{enumerate}
%\item The geometry of a triangular cavity and the position where the leads are attached can be used to selectively couple different pairs of MBS.
%\item Ring shapes strips can reliably connect a single MBS pairs via a single level, but adding a third MBS will split such level into multiple channels each with a narrow potential range.
\item Triangular cavities show a maximum coupling for a certain angle for the far pair, while the coupling of the central pair can be tunned to a maximum or minimum depending on the wire's side.
\item In a gate defined cavity, the position of the nanowires plays a crucial role in the definition and tunability of the triangular shape and thus enhancing or decreasing the MBS coupling.
\item A natural extension of this work is to design an experiment where joint parity measurements can be measured via interferometry in a loop geometry, or via charge measurements with a nearby sensor.
\item Another possible extension is to include realistic noise, representing the etching process, as potential irregularities along the triangle sides.
\item In conclusion, a trijunction of MBS where the coupling of all pairs is comparable to the superconducting gap has been designed and the operation of the device has been discussed in terms of electrostatic gates.
\end{enumerate}