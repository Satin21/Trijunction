\chapter{Conclusions}

In this work we have demonstrated that the question of designing a trijunction of Majorana nanowires is non-trivial.
We have not answered the question of what is the optimal geometry, but we have made progress by analysing different geometries in 1D and 2D.

We have found a systematic difference between different cavity geometries.
In quasi-1D cavities, there is a clear behaviour of the different geometries considered.
On the other hand, 2D systems show a complex behaviour where device size and shape influence the coupling and shape resolution.
Overall, the ring and the triangular geometry are the most interesting candidates in terms of resonance peak hight and width.

The implementation of a gate-defined trijunction requires a systematic tuning of the device.
We have found that the geometric description holds for two-dimensional shapes.
However, as the system size decreases, electrostatic effects play a major role that diverges from the geometrical description.
Interestingly, further optimisation of the coupling can be achieved by controlled the shape of the potential with multiple gates.

This work can be extended in two possible ways:
On the one hand, further geometry analysis can be used to find the optimal cavity shape that will be later implemented with an electrostatic model.
On the other hand, include a measurement protocol that allows to measure the state encoded by the MBS.
This is a necessary ingredient in order to design a Majorana qubit.

The implementation of a gate-defined ring geometry remains as an open question.
The main problem that we found with this geometry is that the coupling between different pairs is highly asymmetric due to resonant trapping.
Nevertheless, this problem can be circumvent by defining multiple regions along the ring that can be activated or depleted depending on the pair to couple.

Another possible extension is to implement an optimisation algorithm that determines the optimal shape of the cavity.
To consider all pairs simultaneously, one can optimise the minimum coupling.
This thesis can be used to establish the constraints on the algorithm, such as device size and nanowire separation.
Once the shape is found, an electrostatic implementation is required to determine the feasibility of such result.

