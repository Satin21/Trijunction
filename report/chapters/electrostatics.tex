\chapter{Gate defined triangular cavities}

\section{Gates configuration}
\begin{enumerate}
\item The triangular cavity is defined using electrostatic gates, and the potential in the 2DEG is found as the solution to the Poisson equation.
\item It is not clear if the geometric dependence holds in a real device where the boundaries of the system are not straight, but smooth following the potential landscape.
\item In contrast to a purely geometric model, changing a single gate has a non-local effect that affects other regions of the potential, possibly inducing unexpected behaviours.
\item The MBS coupling depends on the tradeoff between tunability and shape-resolution determined crucially by the position where the nanowires attach to the cavity.
\item Consider a material stack made by an InAs 2DEG with proximity induced superconductivity, and a set of metallic gates with an oxide layer in between.
\item There are three kinds of gates in this system: plunger gates and screen gates that control the shape of the cavity, and tunnel gates that control the coupling with the nanowires.
\item Devices with three nanowires at one side are larger than those with two because of the minimum separation between tunnel barriers which is required to have well defined coupling channels for each nanowire.
\end{enumerate}

\section{Device operation}
\subsection{Nanowire channels}
\begin{enumerate}
\item In order to have the minimum number of tunnable gates, each nanowire requires a tunnel barrier well separated from each other by fixed-voltage screen gates.
\item The operation point is below the first barrier level resonance in order to avoid interaction with spurious levels and keep a clean cavity dependence.
\item By controlling the tunnel gates height relative to the nanowire's potential, the tunnelling amplitude can be changed from the insulating regime to the strong coupling regime.
\item When the tunnel gates are far from each other, there is no crossed interaction between them, and they can be tunned symmetrically.
\item For closer tunnel gates, there's mutual interaction that modifies the barrier height, center and width, leading to a non-symmetric operational point.
\end{enumerate}
\subsection{Potential deformations}
\begin{enumerate}
\item While the left and right MBS coupling is optimal for a triangular cavity, the coupling of the central pairs is significantly smaller due to the large system size.
\item The triangular shape of the cavity is controlled by three gates, the plunger and the screen side gates, and can be deformed in order to probe modified shapes with increased couplings.
\item The coupling of the central pairs can be significantly increases by detunning the side screen gates and effectively creating smaller triangular cavities.
\item Potential deformations are not allowed in a geometry with the central wire attached to the top triangle vertex because the screen gates determine both the cavity shape and the barrier's positions.
\item Similarly, a configuration with nanowires attached to the diagonal sides would induce an irregularities along these sides that would significantly decrease the MBS coupling.
\end{enumerate}

