\chapter{Background}

\section{Introduction}
\begin{enumerate}
\item Majorana bound states (MBS) can be used to create a non-local qubit robust against local noise.
\item While coupling a single MBS pair can be done using a quantum dot, selective coupling of multiple pairs remains a challenge.
\item In this work we propose a semiconducting cavity connected to three Majorana stripes that allows for an all-electric controlled interaction between all pairs of MBS.
\item Several cavity geometries are analysed, and a triangular cavity with varying angle is found to have the largest coupling for all pairs.
\item The electrostatics effects of the gate-defined triangular cavity are analysed and the operational point is described.
\end{enumerate}

\section{Majorana bound states}

\begin{enumerate}
\item MBS are the non-local degenerate ground state of a topological superconductor as initially proposed by Kitaev.
\item Quantum information can be encoded in the ground state since even and odd parity states are degenerate.
\item Since a pair of spatially separated MBS encodes a single fermionic mode, its quantum state is protected against local errors by particle-hole symmetry.
\end{enumerate}

\section{Experimental platforms}

\begin{enumerate}
\item MBS can be realised in quasi one-dimensional systems defined on two-dimensional electron gases (2DEGs) or semiconducting nanowires in proximity to a superconductor.
% \item A quasi-one dimensional semiconducting system contains multiple transverse sub-bands whose energy and other properties, such as spin-orbit or effective mass, are affected by the geometry and the coupling to the superconductor.
 \item Control over the chemical potential and tunnel coupling to nearby leads is mediated via electrostatic gates, but it is mostly screened inside the nanowire by the presence of the superconductor.
\item Disorder in the nanowire bulk can be detrimental for MBS since it affects its localisation and properties, such as induced gap, while disorder inside the superconductor enhances the induced gap in the nanowire.
\end{enumerate}

\subsection{Two dimensional electron gases}

\begin{enumerate}
\item 2DEGs are a good platform for MBS based quantum computation because complex geometries can be defined in the same layer, whereas nanowires require mechanical matching.
%\item Initial experiments focused on characterising the properties of semiconducting layers, such as large $g$-factor and large spin-orbit, with a superconducting layer.
%\item Later experiments focused on tunnel spectroscopy of stripe-like geometries where a zero bias peak (ZBP) was found, but its origin was trivial Andreev states rather than MBS.
\item 2DEGs are suitable to create gate defined shapes with interesting geometry dependent properties in the ballistic regime, yet in MBS experiments only stripes and planes have been developed so far.
\end{enumerate}


\section{Quantum operations with Majorana bound states}

\begin{enumerate}
%\item The dimension of the computational Hilbert space depends on the number of MBS, and thus multiple parallel Majorana nanowires are required for MBS based quantum computing.
\item The state of the degenerate manifold of MBS can be non-trivially changed by adiabatically moving the particles around each other, i.e. non-Abelian exchange statistics.
\item Initial proposals were based on moving MBS around each other in gate defined nanowire networks, but this method requires high degree of control and is highly susceptible to thermal errors.
\item An equivalent approach that does not require to move the MBS is given by joint parity measurements, but it requires simultaneous measurement of different pairs of MBS.
\item Current proposals for coupling multiple pairs of Majoranas couple each nanowire individually to a QD that later close in a loop.
\item To the best of our knowledge, there is a single study where MBS are connected via a semiconducting cavity in a fork-like geometry, yet geometry role has proven to be fundamental in how leads states couple.
\end{enumerate}

\section{Quantum dot mediated coupling}

\begin{enumerate}
\item A single nanowire coupled to a quantum dot (QD) allows to measure the parity of a pair of MBS via a measurement of the charge in the dot.
%\item In a single nanowire model, the interaction is controlled by the overlap of the MBS pair which depends on the wire length.
\item Two nanowires with a QD in the middle recovers the well-know Josephson junction whose spectra can be controlled by the phase difference between the two superconductors.
\item The fractional Josephson effect, $4\pi$ phase periodicity, is a consequence of the presence of a zero-energy fermionic mode made of two MBS.

\end{enumerate}

